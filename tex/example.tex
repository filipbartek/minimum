\begin{frame}[t]
\frametitle{Reasoning about addition}
\only<1>{Axioms of addition of natural numbers:
\begin{align*}
x + s(y) &= s(x + y) \tag{$\AxiomS$}\\
x + 0 &= x \tag{$\AxiomZ$}
\end{align*}
\note{\begin{itemize}
\item $x, y$ are variables. Each equality is implicitly universally quantified.
\item $+$ is a binary addition function.
\item $s$ is a unary successor function.
\item $0$ is a constant.
\item Natural number $n$ is canonically represented by the term $s^n(0)$.
\item Another axiom we could add: $s(x) \neq 0$
\item $\AxiomS$ and $\AxiomZ$ form equational theory $E$.
$C$ is valid in $E$ because it is satisfied by all the models of $E$.
\end{itemize}}

Conjecture:
\begin{align*}
0 + s(s(0)) &= s(s(0)) \tag{$\Conj$}
\end{align*}

How can we prove that $C$ follows from $\AxiomS$ and $\AxiomZ$?}

\only<2>{Negate the conjecture:
\begin{align*}
0 + s(s(0)) &\neq s(s(0)) \tag{$\NegConj$}\\
x + s(y) &= s(x + y) \tag{$\AxiomS$}\\
x + 0 &= x \tag{$\AxiomZ$}
\end{align*}
Attempt to infer a contradiction.

\note{Proof by contradiction:
\begin{itemize}
\item Assume positive premises and negated conjecture.
\item Infer a trivial contradiction.
\end{itemize}}
}

\only<3-4>{Transform the axioms into rewrite rules:
\begin{align*}
0 + s(s(0)) &\neq s(s(0)) \tag{$\NegConj$}\\
x + s(y) &\to s(x + y) \tag{$\RulePS$}\\
x + 0 &\to x \tag{$\RuleN{0}$}
\end{align*}
\note{This orientation corresponds to LPO($+ > s$) and KBO($+ > s$).}
}

\only<4>{Rewrite $\NegConj$ using $\RulePS$ and $\RuleN{0}$:
\begin{align*}
\underline{0 + s(s(0))} &\neq s(s(0)) &\rewrite{\RulePS}\\
s(\underline{0 + s(0)}) &\neq s(s(0)) &\rewrite{\RulePS}\\
s(s(\underline{0 + 0})) &\neq s(s(0)) &\rewrite{\RuleN{0}}\\
s(s(0)) &\neq s(s(0)) &
\end{align*}}

\only<5-6>{What if we oriented $\AxiomS$ from right to left?
\begin{align*}
0 + s(s(0)) &\neq s(s(0)) \tag{$\NegConj$}\\
x + s(y) &\leftarrow s(x + y) \tag{$\RuleSP$}\\
x + 0 &\to x \tag{$\RuleN{0}$}
\end{align*}

\note{This orientation corresponds to LPO($s > +$) and KBO($s > +$).}
}

\only<6>{Completion procedure yields an infinite sequence of new rules:
\begin{align*}
x + s(0) &\to s(x) \tag{$\RuleN{1}$}\\
x + s(s(0)) &\to s(s(x)) \tag{$\RuleN{2}$}\\
&\vdots\\
x + s^n(0) &\to s^n(x) \tag{$\RuleN{n}$}\\
&\vdots
\end{align*}

\note{Once we have generated $\RuleN{2}$, we can rewrite $\NegConj$ and solve the problem.
A goal-oriented clause selection would favor this inference.}
}
\end{frame}

\begin{frame}
\frametitle{Symbol precedence matters}
\begin{itemize}
\item $\RulePS$ solves the problem deterministically.
\item $\RuleSP$ may lead to an infinite sequence of inferences.
\end{itemize}
\pause
In an \acrfull{atp}, the orientation of equations depends on the \emph{simplification term ordering}
such as the \acrfull{kbo},
which is defined by a \emph{symbol precedence}.
\note{Using LPO yields the same term order.}

\begin{itemize}
\item If $+ > s$, then $x + s(y) >_{\mli{kbo}} s(x + y)$ ($\RulePS$).
\item If $s > +$, then $s(x + y) >_{\mli{kbo}} x + s(y)$ ($\RuleSP$).
\end{itemize}
\end{frame}
