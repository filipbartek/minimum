The state-of-the-art superposition-based theorem provers for \acrlong{fol} rely on simplification orderings on terms to constrain the applicability of inference rules,
which in turn shapes the ensuing search space.
The popular Knuth-Bendix simplification ordering is parameterized by symbol precedence—a permutation of the predicate and function symbols of the input problem’s signature.
Thus, the choice of precedence has an indirect yet often substantial impact on the amount of work required to complete a proof search successfully.

This work describes and evaluates two approaches to the construction of a symbol precedence recommender.
Each of them uses machine learning to estimate the best possible precedence based on observations of prover performance on a set of problems and random precedences.
The first approach uses a small set of simple human-engineered symbol features.
The second approach uses a \acrfull{gcn} to extract meaningful symbol embeddings from the graph structure of the input problem.
When coupled with the theorem prover Vampire and evaluated on the \acrshort{tptp} problem library, the \acrshort{gcn}-based recommender is found to outperform a state-of-the-art heuristic by more than \SI{4}{\percent} on unseen problems.
